\documentclass[a4paper,12pt]{article}
\usepackage{url}
\usepackage{hyperref}
\usepackage{natbib}

\setlength{\oddsidemargin}{0mm}
\setlength{\evensidemargin}{-14mm}
\setlength{\marginparwidth}{0cm}
\setlength{\marginparsep}{0cm}
\setlength{\topmargin}{2mm}
\setlength{\headheight}{0mm}
\setlength{\headsep}{0cm}
\setlength{\textheight}{240mm}
\setlength{\textwidth}{168mm}
\setlength{\topskip}{0mm}
\setlength{\footskip}{10mm}

\newcommand{\code}[1]{\texttt{#1}}
\newcommand{\refsec}[1]{\mbox{Section~\ref{sec:#1}}}
\newcommand{\refapp}[1]{\mbox{Appendix~\ref{sec:#1}}}
\newcommand{\refeqn}[1]{\mbox{(\ref{eqn:#1})}}
\newcommand{\reffig}[1]{\mbox{Figure~\ref{fig:#1}}}
\newcommand{\ud}{\mathrm{d}}                    % upright d (derivative)

\newcounter{foo}
\newcounter{bar}

\bibliographystyle{apalike}
\renewcommand{\bibname}{References}
\renewcommand{\bibsection}{\section{\bibname}}
\renewcommand{\cite}{\citep}


\title{\vspace{-1cm} A guide to writing technical reports}
\author{M. P. Hayes \\
        \small Department of Electrical and Computer Engineering\\
        \small University of Canterbury}
\date{\small Version 4.3}

\begin{document}
\maketitle

\begin{abstract}

This report provides a guide to writing technical reports.  It is
targeted at undergraduate engineering students.  It provides a
recommended structure for reports and makes suggestions on
presentation and writing.


\tableofcontents


\section{Introduction}

Writing your first technical reports is not easy.  The goal of a
technical report is to express findings in a clear and unambiguous
manner.

This report is targeted at undergraduate engineering students to help
them write a technical report.  It is structured as follows.
\refsec{structure} presents a suggested structure for a report.  It
describes the key sections found in a report and provides
recommendations.  This is followed by suggestions for writing style,
grammar, and spelling in \refsec{writing_style}.  Presentation
suggestions are provided in \refsec{presentation} and conclusions are
drawn in \refsec{conclusion}.  Finally, lists of gaucheries and weasel
words are in \refapp{gaucheries} and \refapp{weasel_words}.




\section{Structure}
\label{sec:structure}

Here is a typical top-level structure of a report:
\begin{flushleft}
  Title\\
  Abstract\\
  Table of contents\\
  1 Introduction\\
  2 Background\\
  3 Method\\
  4 Results\\
  5 Discussion\\
  6 Conclusion\\
  7 Acknowledgements (if required)\\
  8 References\\
  A Appendix A\\
  B Appendix B\\
\end{flushleft}
%
The title, abstract, introduction, conclusion, and references are not
optional.  Some reports have no appendices; this example has two.  One
might be for schematics and the other for software listings.


\begin{list}{S-\arabic{foo}}{\usecounter{foo}}

\item All sections starting from the introduction should be numbered.

\item Sections of the report should be referenced by number.  For
  example, \refsec{writing_style} discusses the writing style that
  should be used for a report.  Do not refer to the section above or
  below since they sometimes move around.

\item Do not use too many levels of subsections.  Some styles suggest
  only two levels; three is okay.
\end{list}



\subsection{Abstract}

The abstract is sometimes called an executive summary.  It usually
comes before the table of contents.

The abstract for a short report contains no more than three or four
paragraphs.  Its purpose is to allow the reader to quickly determine
if it is worth reading the rest of the report.

  
\begin{list}{A-\arabic{foo}}{\usecounter{foo}}

\item The abstract is not numbered.

\item The abstract should summarise what is in the report, in particular, it
should specify the key findings.  For example, this report presents
suggestions for how to write a report, \ldots

%\item Avoid too much detail in the abstract.  A reader wants to know
%what the report is about, the major techniques/components used, and
%the outcome of the project.
\end{list}
\end{abstract}


\subsection{Introduction}


This section introduces the purpose of the report and what it to be
found in it.  There is some minor repetition with the abstract but the
introduction should be more detailed.

\begin{list}{I-\arabic{foo}}{\usecounter{foo}}

\item A report must have an introduction.

\item The introduction is the first numbered section.  

\item You should describe the top-level concepts before diving into
  the minut\ae.  A simplified system diagram is useful here.

\item The end of the introduction should contain a \emph{roadmap} to
inform your reader how you have organised your material.  The table of
contents assists a reader who has an idea what they are looking for,
say someone who has already read the report.

For example, \ldots Guidelines for the writing style to be used for a
technical report are given in \refsec{writing_style}.  This is
followed by \ldots
\end{list}


\subsection{Conclusion}


\begin{list}{C-\arabic{foo}}{\usecounter{foo}}
\item All reports should have a conclusion.  No exceptions.  This is
what your boss will read first.

\end{list}




\subsection{References}
\label{sec:referencing}


\begin{list}{R-\arabic{foo}}{\usecounter{foo}}

\item All reports should have a references section.  No, a
  bibliography does not count!  When you make a statement you need to
  be able to back it up.  If the reader thinks you have made a mistake
  or would like to know more details, they can look up the source.
  The references should be numbered\footnote{Personally, I prefer the
    APA style as commonly used for theses where the author's name and
    year is used instead of a number.} and give full details.

\item A reference section is not just a collection of URLs.  These are
  transitory compared to books, papers, and patents.  If you use a URL
  add the date when the web page was last updated.

\item All references must be cited in the report.

\item Use the specified reference style.  The APA style is preferable
  for theses and long reports where there are a large number of
  references.  For short reports, the IEEE transactions style is
  common.
\end{list}


\subsection{Appendices}

\begin{list}{X-\arabic{foo}}{\usecounter{foo}}
\item Software listings belong in an appendix.  Code fragments are
  okay for the main report body to illustrate a point.

\item Detailed tables of experimental results belong in an appendix.
  The main body of the report should only show the general results.

\item Schematics listings belong in an appendix.  The body of the
  report should use block diagrams or simple circuits.

\end{list}






% Rule 1 follow corporate style


\section{Writing}
\label{sec:writing_style}

There are many writing styles.  A technical report should be written
to be concise and unambiguous.  If the reader thinks that your grammar
and spelling is sloppy, they will assume that your arguments are
sloppy.


This section considers the writing tense in \refsec{tense}, followed
by suggestions for sentences in \refsec{sentences}, paragraphs in
\refsec{paragraphs}, punctuation in \refsec{punctuation}, and spelling
in \refsec{spelling}.  It concludes with suggestions for usage of
however and related words in \refsec{however}.  For more suggestions,
refer to the excellent little book called the ``Elements of Style''
\cite{Strunk_1979}.


\subsection{Tense}
\label{sec:tense}

\begin{list}{W-\arabic{foo}}{\usecounter{foo}}

\item A report should be primarily written in the present tense except
  when describing the results of experiments.  In this case, use the
  past tense.

\item There is rarely a need for the future tense.  Rather than saying
  that referencing \emph{will} be considered in \refsec{referencing},
  say that referencing \emph{is} considered in \refsec{referencing}.

\item Ideally reports should be written in the third person although
  it is acceptable in some parts to use the first person, especially
  in an introduction.

\item Only use the first person plural if there are multiple
  authors\footnote{James Clerk Maxwell sometimes added his dog as an
    author so he could use we in a paper instead of I.}.

\item The passive voice should be avoided except when describing
  experimental procedure.

\setcounter{bar}{\value{foo}}
\end{list}


\subsection{Sentences}
\label{sec:sentences}

The key to a technical report is its brevity and lack of ambiguity.
You are not expected to win the reader over by your rhetoric and fancy
sentence composition.

\begin{list}{W-\arabic{foo}}{\usecounter{foo}}
\setcounter{foo}{\value{bar}}
\item Most students write sentences that are too long\footnote{I used
  to.}.  Long sentences are hard for the reader to parse and can
  introduce ambiguity.  

\item There should be only one idea per sentence.

\item Do not join independent clauses with a comma.  Start a new
  sentence.

\item When a sentence is first written it usually contains redundant
  words since this is how we speak.  A competent writer will remove
  the redundant words.  The rule is ``\emph{if you can remove a word
    from a sentence without changing its meaning, then it is not
    needed}''.  For example, ``an instant in time'' is simply an
  ``instant''.  An ``actual fact'' should be replaced with ``fact''.
  See \refapp{gaucheries}.


\item Remove weasel words, see \refapp{weasel_words}.

\setcounter{bar}{\value{foo}}
\end{list}


\subsection{Paragraphs}
\label{sec:paragraphs}

\begin{list}{W-\arabic{foo}}{\usecounter{foo}}
\setcounter{foo}{\value{bar}}
\item Keep paragraphs short.

\item It is unusual to have a single sentence paragraph in a technical
  report\footnote{They are common in newspapers.}.

\item Do not start a paragraph with ``However,'', ``Furthermore,'',
  ``Moreover,'', ``Hence,'', etc.

\setcounter{bar}{\value{foo}}
\end{list}


\subsection{Punctuation}
\label{sec:punctuation}

The correct use of commas can lead to heated debate.  Commas are used
for many things: primarily to add a pause, add a parenthetical phrase
or clause, and to delimit a list.

Delimiting a list is one the most contentious uses of a comma.  I like
the serial comma\footnote{See
  \url{https://en.wikipedia.org/wiki/Serial_comma}.}, aka the Oxford
or Harvard comma.  With the serial comma you would write ``beer, wine,
and spirits'' instead of ``beer, wine and spirits''.  Sentences
without the serial comma are more prone to ambiguity.

\begin{list}{W-\arabic{foo}}{\usecounter{foo}}
\setcounter{foo}{\value{bar}}

\item Whichever comma style you use, the key is to be consistent.
  This way the reader has more confidence that you know what you are
  doing.

\item Novice writers link sentences together with a comma.  Instead,
  it is best to start a new sentence.  If you know what you are doing,
  there are times where you can use a semicolon.

\item Don't use contractions in a technical report!  For example,
  haven't should be have not.

\item The apostrophe indicates possession, i.e., belonging to someone
  or something.  It is not just used to indicate that an s follows at
  the end of the word!  This is a common mistake especially when
  referring to an object by an initialism or acronym.  For example,
  consider a UAV (Unmanned aerial vehicle).  The plural of UAV is
  UAVs, so we would say ``Michael had two UAVs.'' but not ``Michael had
  two UAV's.'' since this violates the use of an apostrophe to indicate
  possession.  However, we would say ``The UAV's propellor was broken.''
  since the propellor belongs to the UAV.

\item When an initialism or acronym is first used it should be
  defined.  For example, the Fast Fourier Transform (FFT) is an
  efficient algorithm for computing a Discrete Fourier Transform
  (DFT).

\item Commas usually follow the abbreviations i.e. and e.g. in the same
  way you would if you said ``that is'' or ``for example''.  

\setcounter{bar}{\value{foo}}
\end{list}


\subsection{Spelling}
\label{sec:spelling}

\begin{list}{W-\arabic{foo}}{\usecounter{foo}}
\setcounter{foo}{\value{bar}}
\item The key is to be consistent and to use the spelling for the person who
commissioned the report.  So for an assignment, that means New Zealand
English.


\item Correct spelling in a report is not optional.  Incorrect spelling
implies that the author is stupid, indolent, or both. 

\setcounter{bar}{\value{foo}}
\end{list}


\subsection{However, furthermore, hence, thus}
\label{sec:however}

Words such as however, furthermore, hence, thus, nevertheless,
etc. are useful for presenting two sides of an argument.  However, it
is bad form to randomly throw them into a sentence.  The best place is
at the start of a sentence, immediately followed by a
comma\footnote{Some people vociferously argue for a semicolon.},
however, they can be used as linking words in the middle of a
sentence.

However you write, be careful with however since it has two meanings!
In this example, however is not followed by a comma.


\section{Presentation}
\label{sec:presentation}


This section considers the presentation of the report.  It considers
recommendations for the main text in \refsec{text}, floats (such as
figures, tables, and listings) in \refsec{figures}, equation
formatting in \refsec{mathematics}, and unit formatting in
\refsec{units}.


\subsection{Text}
\label{sec:text}


\begin{list}{P-\arabic{foo}}{\usecounter{foo}}
\item Use at least a size 11 font for the main text to make it easy to read.

\item Do not make the margins too narrow.  A wide line is hard to scan
  by the human eye.  A rule of thumb is that there should not be more
  than 75 characters per line\footnote{This is why journals and
    newspapers use multiple columns.}.

\item Avoid headings without text between them.  Instead add some
  introductory text to explain what the reader will find in the
  sub-sections.

\item With computer typesetting underlining is frowned upon.  Why?  I
  suppose it does not look good and became unfashionable with the
  demise of the typewriter.  Look at any book.  Instead, use emphasis
  with \textbf{bold} or \textit{italic} fonts.

\item It is a good idea to have program identifiers in a fixed size
font when referred to in the report, for example,
\code{serial\_tx\_byte}.

\item Avoid widowed headings.  These are headings at the bottom of the
  page with no following text.

\item Avoid random capitalisations.  Be consistent.

\end{list}



\subsection{Figures, tables, listings}
\label{sec:figures}

\begin{list}{F-\arabic{foo}}{\usecounter{foo}}
\item While magazines often have text around figures, this is
  considered poor style for a report.  Yes, you can cram in more
  content but does this make the report better?

\item It is important that fonts in figures are not too small.  Be
  careful, when scaling an image since this will alter the size of any
  text.  If you need to reduce the size of an image, generate the text
  in a larger font.

\item It is best to avoid bitmapped images (GIF, BMP, JPEG, etc.) that
contain text since aliasing makes the words hard to read.  If this is
unavoidable, you will need a high resolution of at least 300 dots per
inch.  In comparison, images displayed on web pages have a typical
resolution of only 75 dots per inch.

\item All figures and tables must have captions.  The caption should
  be meaningful and describe the figure/table.  For example, see
  \reffig{example_figure1}.

\item The caption should end with a full stop.  This way you can add
  other sentences and remain consistent with your punctuation.

\item All figures/tables need to be referenced from the text.  For
  example, \reffig{example_figure1} shows the system diagram.

\item All figures and tables must be referred to by an identifier.
  Do not say, ``see the figure below''.  Instead, say ``see
  \reffig{example_figure1}''.

\item It is common to refer to figures (tables) with the word Figure
  (Table) capitalised so that it stands out in the text.  This helps
  the reader to find where the figure (table) is described in the
  text.

\item Software listings should use a fixed size font so that columns
line up.
\end{list}


\begin{figure}
\vspace{1cm}
  \begin{center}
    System diagram to do
\vspace{1cm}
  \end{center}
  \caption{System diagram of the foobar device showing the grand
    frobnicator connected to the little thingamajig at the top.  The
    super grandiloquencer is omitted for clarity.}
\label{fig:example_figure1}
\end{figure}


\subsection{Mathematics}
\label{sec:mathematics}

\begin{list}{M-\arabic{foo}}{\usecounter{foo}}
\item Equations are best separated from the surrounding text.  This
  makes the equations easier to read and allows them to be numbered.

\item To improve the flow of the prose, it is desirable that equations
  are punctuated as part of a sentence.  For example, the voltage
  drop, $V$, across a resistor due to a current, $I$, flowing through
  it is given by Ohm's law,
%
\begin{equation}
  V = R I,
\label{eqn:ohms_law}
\end{equation}
%
where $R$ is the resistance of the resistor.

\item All variables in an equation must be defined when first used.
  The dependent variables are most often defined after the equation.

\item Equations that are referred to should be numbered.  It does not
  hurt to number all equations.

\item When referring to an equation it is desirable to mention it by
  name as well as by its number.  For example, the voltage drop can be
  found using Ohm's law \refeqn{ohms_law}.

\item Is is common to refer to an equation by its number in
  parentheses, for example, \refeqn{ohms_law} instead of
  Equation~\ref{eqn:ohms_law}.

\item Do not use the \verb+*+ symbol for multiplication in an
  equation; use $\times$ or implied multiplication.  For example, $a
  \times b $ or $a b$.

\item Operators and words should be typeset in an upright Roman font,
  not italics.  For example, $\cos$ not $cos$, $\ud x$ not $d x$.

\item Words in superscripts and subscripts should be typeset in an
  upright Roman font, not italics.  For example, $V_{\mathrm{rms}}$,
$I_{\mathrm{max}}$.

\end{list}



\subsection{Units}
\label{sec:units}

\begin{list}{U-\arabic{foo}}{\usecounter{foo}}
\item Technical reports should use SI units.  A unit named after a person is
written in lower case unless it is abbreviated in which case the first
letter is capitalised.    For example, 5\,joules or 5\,J.

\item Prefixes are never capitalised---m means milli but M means mega;
  an error of nine orders of magnitude.  Thus 4\,MHz, not 4\,mhz,
  4~Mhz, etc.

\item The SI standard says units should be separated by a small space
  from the value, for example, 1\,GHz instead of 1GHz.  In
  \LaTeX\ this can be achieved with \verb+1\,GHz+.  Note, you need an
  unbreakable space since you do not want the number and unit on
  separate lines.

\item Units are typeset in an upright Roman font but not italics.  For
  example, 42\,watts not $42\,watts$ and 42\,W not $42\,W$.

\end{list}


\section{Conclusion}
\label{sec:conclusion}

This report provides suggestions for the structure, writing, and
presentation of a technical report.  It is not a representative
example of most technical reports; the recommendations are better
summarised in an appendix.  Nevertheless, hopefully it will help you
write an excellent technical report.



\bibliography{report-guidelines}


\appendix

\section{Gaucheries}
\label{sec:gaucheries}

\newcommand{\gauchery}[2]{#1 $\rightarrow$ #2 \\}

Richard Bates, a former professor of the ECE Department at UC,
insisted that his students expunge all gaucheries from their
writing. He especially hated redundant words and tautologies.  For
example, you were likely to be struck by a thunderbolt if you said an
instant of time.

Here are some gaucheries to expunge from your writing with some
replacements:
\begin{flushleft}
\gauchery{a variety of different}{a variety of}
\gauchery{actual fact}{fact}
\gauchery{adept at}{adept in}
\gauchery{approximation of}{approximation to}
\gauchery{at the conclusion of}{after}
\gauchery{at the present time}{today, at present}
\gauchery{at this point in time}{now}
\gauchery{can be found}{is found}
\gauchery{compensate for}{compensate}
\gauchery{compensating for}{compensating}
\gauchery{connect up}{connect}
\gauchery{correct for}{correct}
\gauchery{correcting for}{correcting}
\gauchery{detailed}{discussed}
\gauchery{doubt but}{doubt}
\gauchery{due to the fact that}{because}
\gauchery{during the course of}{during}
\gauchery{during the time that}{while}
\gauchery{frequently}{often}
\gauchery{help but}{help}
\gauchery{in the course of}{during}
\gauchery{in order to}{to}
\gauchery{in the vicinity of}{about}
\gauchery{in the neighbourhood of}{near}
\gauchery{in this day and age}{today}
\gauchery{instant of time}{instant}
\gauchery{in the event that}{if}
\gauchery{irregardless}{regardless}
\gauchery{join together}{join}
\gauchery{known as}{called}
\gauchery{most unique}{unique}
\gauchery{new and innovative}{innovative}
\gauchery{not got any}{no}
\gauchery{offshore to}{offshore}
\gauchery{prior to}{before}
\gauchery{quite}{}
\gauchery{the reason why}{because}
\gauchery{to begin with}{to begin}
\gauchery{used}{used, applied, employed, invoked, utilised (used is the most overused word in English)}
\gauchery{very}{}
\gauchery{very unique}{unique}
\gauchery{whether or not}{whether}
\end{flushleft}


\section{Weasel words}
\label{sec:weasel_words}

Weasel words\footnote{\url{https://en.wikipedia.org/wiki/Weasel_word}.
See also \citet{Howard_1983}, an interesting book on weasel words.}
should be avoided in a technical report since they obscure clarity.
They are phrases or words that ``sound good'' but do not actually
convey any information.  For example, the word ``actually'' in the
preceding sentence does not add information to the sentence.

There main categories of weasel words are:
%
\begin{enumerate}
\item Vague expressions: some people, experts, it is well known, many,
  somewhat, in most respects, various, fairly, several, extremely,
  exceedingly, quite, remarkably, few, mostly, largely, huge, tiny,
  are a number, excellent, significantly, substantially, clearly,
  vast, relatively, completely, very\footnote{Samuel Clemens (aka Mark
    Twain) said that you should replace all uses of very with damn;
    this way if you do not delete them then your editor will.}.

\item Use of the passive voice to avoid specifying an authority: it is
  said.

\item Adverbs that weaken: often, probably, possibly, truly, actually,
  basically, interestingly, surprisingly.
\end{enumerate}


\section{Other resources}

\begin{itemize}
\item Grammarly is a free online English grammar and spelling checker.
  See \url{/www.grammarly.com}.
\end{itemize}


\end{document}

